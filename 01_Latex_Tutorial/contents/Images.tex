%%%%%%%%%%%%%%%%%%%%%%%%%%%%%%%%%%%%%%%%%%%%%
%
% $Autor: Wojciech Zarzycki $
% $Datum: 2020-09-17 12:27:00Z $
% !TEX root = ../LatexTutorial.tex
% $Version: 1 $
%
%%%%%%%%%%%%%%%%%%%%%%%%%%%%%%%%%%%%%%%%%%%%%

\begin{tcolorbox}[red,title={Hinweis}]
Środowisko figure to tzw. float - czyli obiekt pływający. Dzięki temu wszystkie grafiki są sformatowane zgodnie z normą, kada figura dostaje swój numer i poźniej jest prezentowana w liście rysunków.
\end{tcolorbox}


\section{Add Image}
\begin{lstlisting}
\begin{figure}[H] % oder [htbp]
    \centering
    \includegraphics[width=0.7\linewidth]{path}
    \caption{title}
    \label{fig:label}
\end{figure}

\end{lstlisting}

\section{Image and Tikz}
\begin{lstlisting}
\begin{figure}[H] % oder [htbp], je nach Platzierungswunsch
  \centering
  \begin{tikzpicture}
    % obrazek jako tlo
    \node[anchor=south west, inner sep=0] (image) at (0,0) {%
      \includegraphics[width=0.7\linewidth]{path}};
        
    \begin{scope}[x={(image.south east)}, y={(image.north west)}]
      \draw[red, line width=4pt, <-, >=angle 60] (0.37,0.755) -- ++(0.2,-0.1);
        
      % lib_deps
      \draw[red, line width=2pt] (0.06,0.52) rectangle ++(0.5, 0.08);
        
      % #include <XXXX.h>
      \draw[red, line width=2pt] (0.06,0.32) rectangle ++(0.35, 0.08);
    \end{scope}
  \end{tikzpicture}
  \caption{Title}
  \label{fig:label}
\end{figure}
\end{lstlisting}

\begin{tcolorbox}[blue,title={Hinweis}]
Argument opcjonalny [htbp] (here, top, bottom, page) pozwala LaTeXowi decydować, gdzie najlepiej umieścić rysunek, aby układ był estetyczny:
    \begin{itemize}
        \setlength\topsep{0em}      
        \setlength\itemsep{0.2em}   % odstęp między punktami
        \setlength\parskip{0pt}     % odstęp między akapitami
        \setlength\parsep{0pt}
        
        \item \texttt{[h]} → ,,here'' - jak najbliżej miejsca w kodzie
        \item \texttt{[t]} → ,,top'' - na górze strony
        \item \texttt{[b]} → ,,bottom'' - na dole strony
        \item \texttt{[p]} → osobna strona tylko z rysunkami
    \end{itemize}
Zwykle używa się \texttt{[htbp]}, a dla wymuszenia \texttt{[H]} (wymaga \texttt{\textbackslash usepackage\{float\}}).

\end{tcolorbox}
