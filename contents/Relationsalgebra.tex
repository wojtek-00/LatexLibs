% filename.....
% !TeX root = ../main.tex
%%%%%

\section{Relationsale Algebra}

Relationale Algebra to zbiór operacji, które przyjmują jedną lub więcej relacji (tabel) jako dane wejściowe i zwracają nową relację jako wynik.
Wszystko w niej opiera się na zbiorach i operacjach matematycznych.


\vspace{0.5cm}
Die wichtigste operationen:
\begin{table}[H]
\centering
\caption{Wichtige Operationen der Relationalen Algebra}
\begin{tabularx}{\textwidth}{|l|c|X|}
\hline
\textbf{Operation} & \textbf{Symbol} & \textbf{Beschreibung} \\
\hline
Selektion & $\sigma$ & Wählt Tupel aus, die eine gegebene Bedingung erfüllen. \\
\hline
Projektion & $\pi$ & Wählt bestimmte Attribute (Spalten) einer Relation aus. \\
\hline
Vereinigung & $\cup$ & Kombiniert die Tupel zweier Relationen mit gleicher Struktur (wie \texttt{UNION} in SQL). \\
\hline
Differenz & $-$ & Liefert die Tupel, die in der ersten, aber nicht in der zweiten Relation vorkommen. \\
\hline
Kartesisches Produkt & $\times$ & Bildet alle möglichen Kombinationen von Tupeln aus zwei Relationen. \\
\hline
Join (Verbund) & $\bowtie$ & Verknüpft zwei Relationen über gleiche Attribute oder Bedingungen. \\
\hline
\end{tabularx}
\end{table}

\subsection{Relationen}
Eine Relation R (Tabelle) ist eine Teilmenge des Kreuzproduktes $Att_1 \times ... \times Att_n$. Dies wird $R \subseteq Att_1 \times ... \times Att_n$ geschrieben.

\begin{tcolorbox}[red, title={Hinweis}]
	Relacja to mattematyczny model tabeli w relacynej bazie danych. Jest to zbiór tuples, które mają taką samą struktórę atrybutów:\\
    $R \subseteq Att_1 \times ... \times Att_n$\\

    gdzie:
    \begin{enumerate}[label={}, leftmargin=15pt, itemsep=0pt, topsep=0pt]
        \item R - nazwa relacji (np. Student)
        \item $A_1, A_2, ..., A_n$ nazy atrybutów(Matrikellnummer, Name, Fachrichtung).
    \end{enumerate}
\end{tcolorbox}

\subsection{Vereinigungsverträglichkeit}
2 Relationen sind Vereinigungsverträglich, wenn sie:
\begin{enumerate}
    \item denselben Anzahl an Attributen haben
    \item die entsprechenden Attribute $A_i$ in R und $B_i$ in S denselben Datentyp oder einen gemeinsamen Obertyp besitzen.
\end{enumerate}



\begin{tcolorbox}[red, title={Hinweis}]
    \begin{align*}
        R&\subseteq Att_1 \times \dots \times Att_n\\
        S&\subseteq Btt_1 \times \dots \times Btt_n\\
        &mit Typ(A_i) = Typ(B_i)
    \end{align*}
\end{tcolorbox}

\subsubsection{Beispiel}
\begin{table}[H]
\centering
\caption{Beispiel: Vereinigungsverträgliche Relationen}
\begin{minipage}{0.3\textwidth}
\centering
\textbf{Relation R} \\[3pt]
\begin{tabularx}{\linewidth}{|X|X|}
\hline
Matr-nummer & Name \\
\hline
101 & Anna \\
102 & Ben \\
\hline
\end{tabularx}
\end{minipage}
\hfill
\begin{minipage}{0.3\textwidth}
\centering
\textbf{Relation S} \\[3pt]
\begin{tabularx}{\linewidth}{|X|X|}
\hline
Matr-nummer & Name \\
\hline
103 & Carla \\
104 & David \\
\hline
\end{tabularx}
\end{minipage}
\hfill
\begin{minipage}{0.3\textwidth}
\centering
\textbf{Relation C} \\[3pt]
\begin{tabularx}{\linewidth}{|X|X|}
\hline
Name & Alter \\
\hline
Anna & 21 \\
Ben & 22 \\
\hline
\end{tabularx}
\end{minipage}
\end{table}

Die relationen R und S sind verträglich, da Sie in einer tabelle dargestellt werden (verbunden). Die relationen R und C oder S und C sind net kompatibel, da die Attribute nicht gleich sind (siehe~\autoref{relOperations})

\newpage
\subsection{Mengenoperationen für Relationen \label{relOperations}}
Seien R und S vereinigungsverträglich, dann kann man neue Relationen berechnen:
\begin{enumerate}
    \item \textbf{Schnittmenge} $R \cap S = \{\, r \mid r \in R \land r \in S \,\}$ - Einträge, die in beiden Relationen vorkommen
    \item \textbf{Vereiguung} $R \cup S = \{\, r \mid r \in R \lor r \in S \,\}$ - Zusammenfassung aller Einträge der Relationen
    \item \textbf{Differenz} $R - S = \{\, r \mid r \in R \land r \notin S \,\}$ - Suchen nach EInträgen, die nur in der ersten, aber nicht in der zweiten Relation vorkommen
\end{enumerate}
\begin{tcolorbox}[red, title={Hinweis}]
    Bei Relationen handelt es sich um Mengen, daher keine Zeile kommt doppelt vor!
\end{tcolorbox}

\subsubsection{Beispiel}
\begin{table}[H]
    \centering
    \begin{minipage}{0.4\textwidth}
    \centering
    \textbf{VK} \\[3pt]
    \begin{tabularx}{\linewidth}{|X|X|X|}
    \hline
    Verkäufer & Produkt & Käufer \\
    \hline
    Meier & Hose & Schmidt \\
    Müller & Rock & Schmidt \\
    Meier & Hose & Schulz \\
    \hline
    \end{tabularx}
    \end{minipage}
    \hfill
    \begin{minipage}{0.4\textwidth}
    \centering
    \textbf{VK2} \\[3pt]
    \begin{tabularx}{\linewidth}{|X|X|X|}
    \hline
    Verkäufer & Produkt & Käufer \\
    \hline
    Müller & Hemd & Schmidt \\
    Müller & Rock & Schmidt \\
    Meier & Rock & Schulz \\
    \hline
    \end{tabularx}
    \end{minipage}
\end{table}


\begin{table}[H]
    \centering
    \begin{minipage}{0.4\textwidth}
    \centering
    \textbf{VK $\cup$ VK2} \\[3pt]
    \begin{tabularx}{\linewidth}{|X|X|X|}
    \hline
    Verkäufer & Produkt & Käufer \\
    \hline
    Meier & Hose & Schmidt \\
    Müller & Rock & Schmidt \\
    Meier & Hose & Schulz \\
    Müller & Hemd & Schmidt \\
    Meier & Rock & Schulz \\
    \hline
    \end{tabularx}
    \end{minipage}
    \hfill
    \begin{minipage}{0.4\textwidth}
    \centering
    \textbf{VK $\cap$ VK2} \\[3pt]
    \begin{tabularx}{\linewidth}{|X|X|X|}
    \hline
    Verkäufer & Produkt & Käufer \\
    \hline
    Müller & Rock & Schmidt \\
    \hline
    \end{tabularx}
    \vspace{12pt} % odstęp pionowy między tabelkami

    \centering
    \textbf{VK - VK2} \\[3pt]
    \begin{tabularx}{\linewidth}{|X|X|X|}
    \hline
    Verkäufer & Produkt & Käufer \\
    \hline
    Meier & Hose & Schmidt \\
    Meier & Hose & Schulz \\
    \hline
    \end{tabularx}
    \end{minipage}
\end{table}

\newpage
\subsection{Projektion}
Sei $R \subseteq Att_1 \times \dots \times Att_n$ eine Relation und $B_1, \dots, B_j$ verschiedene Attribute aus der Menge $\{Att_1, \dots, Att_n\}$. 

Dann ist die \textbf{Projektion} von $R$ auf $B_1, \dots, B_j$, geschrieben als
\[
\text{Proj}(R, [B_1, \dots, B_j]),
\]
die Relation, die entsteht, wenn man aus $R$ alle Spalten entfernt, die nicht in $B_1, \dots, B_j$ enthalten sind. 

Die Reihenfolge der Attribute $B_1, \dots, B_j$ bestimmt zugleich die Reihenfolge der Spalten in der Ergebnisrelation.

\begin{tcolorbox}[red, title={Hinweis}]
    Projekcja służy do wyboru określonych kolumn (atrybutów) z relacji. Odrzuca wszystkie pozostałe atrybuty i często też usuwa duplikaty, ponieważ relacja w matematycznym sensie to zbiór (a zbiór nie zawiera powtórzeń).
\end{tcolorbox}

\subsubsection{Beispiel}

\begin{table}[H]
    \centering
    \begin{minipage}{0.4\textwidth}
    \centering
    \textbf{$\text{Proj}(VK, [Käufer, Product])$} \\[3pt]
    \begin{tabularx}{\linewidth}{|X|X|}
    \hline
    Käufer & Produkt \\
    \hline
    Schmidt & Hose \\
    Schmidt & Rock \\
    Schulz & Hose \\
    \hline
    \end{tabularx}
    \end{minipage}
    \hfill
    \begin{minipage}{0.4\textwidth}
    \centering
    \textbf{$\text{Proj}(VK, [Verkäufer])$} \\[3pt]
    \begin{tabular}{|c|}
    \hline
    Verkäufer \\
    \hline
    Meier \\
    Müller \\
    \hline
    \end{tabular}
    \vspace{12pt} % odstęp pionowy między tabelkami

    \centering
    \textbf{$\text{Proj}(VK \cap VK2, [Produkt])$} \\[3pt]
    \begin{tabular}{|c|}
    \hline
    Produkt \\
    \hline
    Rock \\
    \hline
    \end{tabular}
    \end{minipage}
\end{table}

\subsection{Umbennenung (Rename)}
Sei R eine Relation, dann bezeichnet $Ren(R,\,T)$ eine Relation mit gleichem Inhalt wie R, die T gennant wird. 

\begin{tcolorbox}[red, title={Hinweis}]
    Tego używa się gdy tabela sama ze sobą musi być zestawiona. Jeśli mamy 2 razy nazwę tej samej tabeli i potem chcemy operowac na Atrybutach tej tabeli to SQL nie wie o ktorą tabekę nam chodzi. Dlatego robimy TAB1 i $Ren(TAB1, TAB2)$ i terazpod TAB1 i TAB2 mamy tą samą tabelę i moeby operowac na jej kolumnach
\end{tcolorbox}

\subsection{Auswahl (Select)}
Sei R eine Relation, dann bezeichnet $Sel(R,\, Bed)$ eine Relation, die alle Zeilen aus R beinhaltet, die die Bedingung Bed erfüllen.

\begin{tcolorbox}[blue, title={Syntax}]
Syntax der Bedingungen $Bed$: $Att_1\ \ OPERATOR\ \ KONSTANTE$
\begin{itemize}
    \item OPERATOR - =, <>, <=, >=, <, >
    \item KONSTANTE - muss ein Wert des zum Attribut gehörenden Datentyps sein. Es kann auch ein Attribut aus anderer Spalte sein - hierbei muss der Typ des Attibunts gleich sein, oder sie müssen einen Gemeinsamen Obertyp besitzen
\end{itemize}
\end{tcolorbox}

Es besteht auch die möglichkeit mehrere Bedingungen einzuführen:
\begin{itemize}
    \item $Bed_1\ \  AND\ \ Bed_2$ - beide Bedingungen sollen erfüllt sein
    \item $Bed_1\ \  OR\ \ Bed_2$ - mindenstens eine der Bedingungen soll erfüllt sein
    \item $NOT\ \ Bed_1$ - die Bedingung soll nicht erfüllt sein
    \item $(Bed_1)$ - die Bedingung in Klammern werden zuerst ausgewertet
\end{itemize}

\vspace{0.5cm}
\subsubsection{Beispiel: Alle Berkäufe, die Meier gemacht hat}
$Sel(VK,\ VK.Verkäufer = \text{'Meier'})$
\begin{table}[H]
    \centering
    \begin{tabular}{|c|c|c|}
        \hline
        \textbf{Verkäufer} & \textbf{Produkt} & \textbf{Käufer} \\
        \hline
        Meier & Hose & Schmidt \\
        \hline
        Meier & Hose & Schluz \\
        \hline
    \end{tabular}
\end{table}

\subsubsection{Beispiel: Alle Käufer, die bei Meier gekauft haben}
$Proj(Sel(VK,\ VK.Verkäufer = \text{'Meier}), [\text{'Käufer'}])$

\begin{table}[H]
    \centering
    \begin{tabular}{|c|}
        \hline
        \textbf{Käufer} \\
        \hline
        Schmidt \\
        \hline
        Schluz \\
        \hline
    \end{tabular}
\end{table}

\subsubsection{Beispiel: Alle Verkäufr, die Meier gemacht hat und die nicht den Kunden Schulz betreffen}

\section{Ein Verknüpfungsoperator für Relationen}
Bislang beziehen sich operationen auf einzelne Tabellen. Durch das kreuzprodukt können mehrerer, auch verschiedene Tabellen miteinander Verknüpft.

\subsection{Verknüpfung von Tupeln (Konkatenation)}
Seien \textit{R} und \textit{S} Relationen mit $r = \{r_1, \dots, r_n\} \in R$ und $s = \{s_1, \dots, s_n\} \in S$. Dann ist die Verknüpfung oder Konkatenation von \textit{r} mit \textit{s}, geschrieben $r \circ s$, definiert als $\{r_1, \dots, r_n, s_1, \dots, s_n\}$.

\subsection{Kreuzprodukt}
Seien \textit{R} und \textit{S} Relationen, dann ist das kreuzprodukt von \textit{R} und \textit{S}, geschrieben $R \times S$, sefiniert durch $R \times S = \{r \circ s | r \in R \text{ und } s \in S\}$

\begin{tcolorbox}[blue, title={Hinweis}]
   Konkatenacja to operacja na pojedynczych elementach, łączy je w jeden dłuszy element. Kreuzprodukt to operacja na zbiorach elementów, która generuje nową relację, która zawiera wszystkie moliwe kombinacje krotek z $R \text{ i } S$
   
\end{tcolorbox}

\begin{tcolorbox}[red, title={Konkatenation vs Kreuzprodukt}]
Seien 
\[
R = \{(a_1), (a_2)\} \quad \text{und} \quad S = \{(b_1), (b_2)\}.
\]

Dann ist die \textbf{Konkatenation} einzelner Tupel definiert als:
\[
(a_1) \circ (b_1) = (a_1, b_1)
\]

Das \textbf{Kreuzprodukt} der Relationen $R$ und $S$ besteht aus allen möglichen Konkatenationen von Tupeln aus $R$ und $S$:
\[
R \times S = \{\, r \circ s \mid r \in R,\, s \in S \,\}
\]

Konkret ergibt sich hier:
\[
R \times S = \{(a_1, b_1), (a_1, b_2), (a_2, b_1), (a_2, b_2)\}
\]
\end{tcolorbox}

\subsection{Übung}
\begin{table}[H]
    \centering
    \textbf{Relationen} \\ [10pt]
    \begin{minipage}[t]{0.3\textwidth}
        \centering
        \textbf{Projekt} \\[3pt]
        \begin{tabularx}{\linewidth}{|X|X|}
            \hline
            \textbf{ProNr} & \textbf{Name} \\
            \hline
            1. & Schachtel \\
            \hline
            2. & Behang \\
            \hline
        \end{tabularx}
    \end{minipage}
    \hspace{0.3cm}
    \begin{minipage}[t]{0.3\textwidth}
        \centering
        \textbf{Aufgabe} \\ [3pt]

        \begin{tabularx}{\linewidth}{|X|X|X|}
            \hline
            \textbf{AufNr} & \textbf{Arbeit} & \textbf{ProNr} \\
            \hline
            1. & knicken    & 1 \\
            2. & kleben     & 1 \\
            3. & knicken    & 2\\
            4. & färben     & 2 \\
            \hline
        \end{tabularx}
    \end{minipage}
    \hspace{0.3cm}
    \begin{minipage}[t]{0.3\textwidth}
        \centering
        \textbf{Maschine} \\ [3pt]

        \begin{tabularx}{\linewidth}{|X|X|X|}
            \hline
            \textbf{Mname} & \textbf{Dater} & \textbf{AufNr} \\
            \hline
            M1 & 2    & 1 \\
            M2 & 3    & 1 \\
            M1 & 3    & 2\\
            M3 & 2    & 3 \\
            M1 & 1    & 4 \\
            M4 & 3    & 4 \\
            \hline
        \end{tabularx}
    \end{minipage}
\end{table}

\begin{enumerate}
    \item Geben Sie die Namen aller möglichen Arbeiten an \\ \[Proj(Aufgabe, [Arbeit])\]
    \item Geben Sie zu jedem Projektnamen die zugehörigen Arbeiten an. Das Ergebnis ist eine Relation mit den Attributen ,,Name'' und ,,Arbeit''. 
                \[
                    \text{Proj}\bigl(
                    \text{Sel}(Projekt \times Aufgabe,\, Projekt.ProNr = Aufgabe.ProNr),
                    [\text{Name, Arbeit}]
                    \bigr)
                    \]
            \begin{tcolorbox}[blue, title={Przykład dla Kreuzprodukt $Projekt \times Aufgabe$}]
                \begin{table}[H]
                    \centering
                    \textbf{$Projekt \times Aufgabe$} \\[6pt]
                    \begin{tabular}{|c|l|c|l|c|}
                        \hline
                        \textbf{ProNr} & \textbf{Name} & \textbf{AufNr} & \textbf{Arbeit} & \textbf{ProNr} \\
                        \hline
                        1 & Schachtel & 1 & knicken & 1 \\
                        1 & Schachtel & 2 & kleben  & 1 \\
                        1 & Schachtel & 3 & knicken & 2 \\
                        1 & Schachtel & 4 & färben  & 2 \\
                        2 & Behang    & 1 & knicken & 1 \\
                        2 & Behang    & 2 & kleben  & 1 \\
                        2 & Behang    & 3 & knicken & 2 \\
                        2 & Behang    & 4 & färben  & 2 \\
                        \hline
                    \end{tabular}
                \end{table}
            \end{tcolorbox}
    \item Welche Maschinen werden zum Knicken genutzt?
                \begin{align*}
                    \text{Proj}\bigl(
                    \text{Sel}(&Aufgabe \times Maschine,\,
                    Aufgabe.AufNr = Maschine.AufNr \\
                    & \text{AND } Aufgabe.Arbeit = \text{'knicken'}),\,
                    [\text{Mname}]
                    \bigr)
                \end{align*}

    \item Geben Sie zu jedem Projektnamen die Maschinen aus, die genutzt werden
        \begin{align*}
            \text{Proj}\bigl(
            \text{Sel}&(Projekt \times Aufgabe \times Maschine, Projekt.ProNr = Aufgabe.ProNr \\ 
            &\text{ AND } Aufgabe.AufNr = Maschine.AufNr),\, [ProjName, Mname]
            \bigr)
        \end{align*}

    \item Geben Sie alle Projekte (deren Namen) aus, bei denen geknickt und gefärbt wird
        \begin{align*}
            \text{Proj}\bigl(
            \text{Sel}&(Projekt \times Aufgabe \times \text{Ren}(Aufgabe, A2),\, 
            Projekt.ProNr = Aufgabe.ProNr \text{ AND } \\
            &Projekt.ProNr = A2.ProNr \text{ AND } \\
            &Aufgabe.Arbeit = \text{'knicken'} \text{ AND } A2.Arbeit = \text{'färben'}),\, 
            [\text{Name}]
            \bigr)
        \end{align*}
\end{enumerate}
