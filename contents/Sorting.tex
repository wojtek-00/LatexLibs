% filename.....
% !TeX root = ../main.tex
%%%%%

\section{Sorting}
\begin{tcolorbox}[gray, title={\rCode{compareTo(T o)}}]

    Słuy do porównywania obiektór tego samego typu. Zwraca:
    \begin{itemize}
        \item liczbę ujemną jesli \rCode{this < other}
        \item \rCode{0} jesli \rCode{this == other}
        \item liczbę dodatnią jesli \rCode{this > other}
    \end{itemize}
    Działa wewnątrz klasy i pochodzi z \rCode{Comparable}.

   \begin{java}
        public class Student implements Comparable<Student> {
        private String name;
        private int age;

        public Student(String name, int age) {
            this.name = name;
            this.age = age;
        }

        @Override
        public int compareTo(Student other) {
            return Integer.compare(this.age, other.age);
        }

        @Override
        public String toString() {
            return name + " (" + age + ")";
        }
    }

    public class Main {
        public static void main(String[] args) {
            List<Student> students = Arrays.asList(
                new Student("Anna", 22),
                new Student("Kuba", 20),
                new Student("Ola", 25)
            );

            Collections.sort(students); // uses compareTo()
            System.out.println(students);
        }
    }

    \end{java}
\end{tcolorbox}


\begin{tcolorbox}[gray, title={\rCode{comparator(T o)}}]
    Świetne pytanie — Comparator robi to samo co Comparable, czyli porównuje obiekty, ale działa z zewnątrz, a nie wewnątrz klasy.
    \begin{table}[H]
        \centering
        \small
        \begin{tabularx}{0.95\textwidth}{|X|X|X|}
            \hline
            \textbf{Cecha} & \textbf{Comparable} & \textbf{Comparator} \\
            \hline
            Gdzie się definiuje & w klasie obiektu \rCode{implements Comparable} & w osobnej klasie lub lambdzie \\
            \hline
            Liczba moliwych porównań & Tylko jedno & dowolna liczba \\
            \hline
            Metoda & \rCode{compareTo(T other)} & \rCode{compare(T o1, T o2)} \\
            \hline
        \end{tabularx}
    \end{table}

    \pheading{Przykład}
    Załóżmy, że klasa \rCode{Student} nie implementuje \rCode{Comparable}:
    \begin{java}
        public class Student {
            String name;
            int age;

            public Student(String name, int age) {
                this.name = name;
                this.age = age;
            }

            @Override
            public String toString() {
                return name + " (" + age + ")";
            }
        }
    \end{java}

    Teraz tworzymy Comparator:
    \begin{java}
        import java.util.*;

        public class Main {
            public static void main(String[] args) {
                List<Student> students = Arrays.asList(
                    new Student("Anna", 22),
                    new Student("Kuba", 20),
                    new Student("Ola", 25)
                );

                // Comparator po wieku
                Comparator<Student> byAge = (s1, s2) -> Integer.compare(s1.age, s2.age);
                Collections.sort(students, byAge);

                System.out.println(students);
            }
        }
    \end{java}

    Wynik to: \rCode{[Kuba (20), Anna (22), Ola (25)]
}

\end{tcolorbox}



\begin{tcolorbox}[red, title={Porównywanie obiektów w~Java: \rCode{==} vs \rCode{equals()}}]

    \begin{itemize}
    \item \rCode{==}~– porównuje \textbf{referencje obiektów}, czyli sprawdza, 
    czy dwie zmienne wskazują na \emph{ten sam obiekt w pamięci}.
    \[
        a == b \;\Rightarrow\; \text{czy } a \text{ i } b \text{ to ten sam obiekt?}
    \]

    \item \rCode{equals()}~– porównuje \textbf{zawartość obiektów}, 
    czyli sprawdza, czy dane przechowywane w obiektach są takie same.
    \[
        a.equals(b) \;\Rightarrow\; \text{czy } a \text{ i } b \text{ mają te same dane?}
    \]

    \item Domyślna implementacja \rCode{equals()} w~klasie \rCode{Object} działa 
    tak samo jak \rCode{==}. Aby porównywać zawartość, należy ją \textbf{nadpisać}:
    \end{itemize}

    \begin{java}
    @Override
    public boolean equals(Object o) {
        if (this == o) return true;
        if (!(o instanceof Student)) return false;
        Student s = (Student) o;
        return name.equals(s.name);
    }
    \end{java}

    \noindent
    \textbf{Podsumowanie:}

    \begin{center}
    \begin{tabular}{|l|l|l|}
    \hline
    \textbf{Operator} & \textbf{Porównuje} & \textbf{Użycie} \\
    \hline
    \rCode{==} & Referencję (adres pamięci) & Tożsamość obiektu \\
    \rCode{equals()} & Zawartość (dane) & Równość logiczna \\
    \hline
    \end{tabular}
    \end{center}

\end{tcolorbox}
