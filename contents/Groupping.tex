% filename.....
% !TeX root = ../main.tex
%%%%%

\section{\rCode{GROUP BY}, \rCode{HAVING} i \rCode{ORDER BY} w SQL}

\subsection{Motywacja}
    Zwykła klauzula \rCode{WHERE} pozwala filtrować pojedyncze wiersze, ale nie umożliwia tworzenia zbiorczych zestawień. 
 
    \begin{sql}
    SELECT COUNT(*) AS Schafanzahl
        FROM Tier
        WHERE Tier.Gattung = 'Schaf';
    \end{sql}

    Takie rozwiązanie jest niewygodne, ponieważ dla każdej wartości (\rCode{Gattung}) trzeba wykonać osobne zapytanie.  
    Dzięki klauzuli \rCode{GROUP BY} można uzyskać podsumowanie dla wszystkich grup w jednym zapytaniu.

\subsection{Podstawowa idea grupowania}
    \rCode{GROUP BY} dzieli dane w tabeli na grupy wierszy, które mają te same wartości określonych kolumn.  
    Dla każdej grupy można następnie obliczyć wartości przy użyciu funkcji agregujących takich jak \rCode{COUNT}, \rCode{SUM}, \rCode{AVG}, \rCode{MIN} czy \rCode{MAX}.

    \begin{sql}
    SELECT Tier.Gattung, COUNT(*) AS Tieranzahl
        FROM Tier
        GROUP BY Tier.Gattung;
    \end{sql}

\paragraph{Etapy działania}
\begin{enumerate}
    \item Posortowanie danych według kolumny \rCode{Gattung}:
    \begin{sql}
    SELECT * FROM Tier ORDER BY Tier.Gattung;
    \end{sql}
    \item Utworzenie grup dla każdej wartości atrybutu (np. \rCode{Bär}, \rCode{Hase}, \rCode{Schaf})
    \item Zastosowanie funkcji agregującej \rCode{COUNT(*)} dla każdej grupy
\end{enumerate}

\subsection{Zasady stosowania \rCode{GROUP BY}}
    W klauzuli \rCode{SELECT} mogą znajdować się wyłącznie:
    \begin{itemize}
        \item kolumny wymienione w \rCode{GROUP BY},
        \item lub funkcje agregujące.
    \end{itemize}

    \begin{sql}
    SELECT Tier.GNr, Tier.Gattung, COUNT(*) AS Tieranzahl
    FROM Tier
    GROUP BY Tier.GNr, Tier.Gattung;
    \end{sql}

\subsection{Klauzula \rCode{HAVING}}
    Klauzula \rCode{HAVING} służy do filtrowania \textbf{całych grup}, w przeciwieństwie do \rCode{WHERE}, która filtruje pojedyncze wiersze.  
    Może zawierać warunki z funkcjami agregującymi.

\begin{tcolorbox}[gray]
    \textbf{Przykład 1:} Numery wybiegów, w których znajduje się co najmniej 3 zwierzęta.
    \begin{sql}
    SELECT Tier.GNr
        FROM Tier
        GROUP BY Tier.GNr
        HAVING COUNT(*) >= 3;
    \end{sql}
\end{tcolorbox}

\begin{tcolorbox}[gray]
    \textbf{Przykład 2:} Gatunki inne niż \rCode{Schaf}, w których liczba zwierząt nie przekracza 3.
    \begin{sql}
    SELECT Tier.Gattung, COUNT(*) AS Tieranzahl
        FROM Tier
        GROUP BY Tier.Gattung
        HAVING Tier.Gattung <> 'Schaf' AND COUNT(*) <= 3;
    \end{sql}
\end{tcolorbox}

\subsection{Klauzula \rCode{ORDER BY}}
    Po grupowaniu można uporządkować wynik według określonego kryterium, np. liczby zwierząt:
    \begin{sql}
    SELECT Tier.Gattung, COUNT(*) AS Tieranzahl
        FROM Tier
        GROUP BY Tier.Gattung
        ORDER BY COUNT(*) DESC;
    \end{sql}

\subsection{Grupowanie przy wielu tabelach}
    \rCode{GROUP BY} może być stosowane również po połączeniu tabel przy użyciu \rCode{JOIN} lub warunków w \rCode{WHERE}.

    \begin{tcolorbox}[gray]
        \paragraph{Przykład:} Powierzchnia zajmowana przez zwierzęta w każdym wybiegu.
        \begin{sql}
        SELECT Gehege.GName, SUM(Art.MinFlaeche) AS Verbraucht
            FROM Gehege, Tier, Art
            WHERE Gehege.GNr = Tier.GNr
            AND Tier.Gattung = Art.Gattung
            GROUP BY Gehege.GName;
        \end{sql}
    \end{tcolorbox}

    \begin{tcolorbox}[gray]
        \paragraph{Przykład:} Powierzchnia (rosnąco), którą zajmuje każdy gatunek, pod warunkiem, że w zoo znajduje się co najmniej 3 przedstawicieli tego gatunku:
        \begin{sql}
        SELECT Tier.Gattung, SUM(Art.MinFlaeche) AS Bedarf
            FROM Tier, Art
            WHERE Tier.Gattung = Art.Gattung
            GROUP BY Tier.Gattung
            HAVING COUNT(*) >= 3
            ORDER BY SUM(Art.MinFlaeche);
        \end{sql}
    \end{tcolorbox}

\subsection{Kolejność wykonywania zapytania SQL}
    \begin{table}[H]
        \centering
        \scriptsize
        \begin{tabular}{|c|c|p{7cm}|}
            \hline
            \textbf{Klauzula} & \textbf{Kolejność} & \textbf{Opis} \\
            \hline
            \rCode{FROM} & 1 & Wybór tabel (tworzenie iloczynu kartezjańskiego) \\
            \rCode{WHERE} & 2 & Filtrowanie wierszy przed grupowaniem \\
            \rCode{GROUP BY} & 3 & Tworzenie grup \\
            \rCode{HAVING} & 4 & Filtrowanie grup \\
            \rCode{ORDER BY} & 5 & Sortowanie wyników końcowych \\
            \rCode{SELECT} & 6 & Wybór kolumn i funkcji agregujących do wyświetlenia \\
            \hline
        \end{tabular}
    \end{table}

\subsection{Strategia pisania zapytań z grupowaniem}
\begin{enumerate}
    \item Wybierz tabele i wpisz je w \rCode{FROM}.
    \item Zdefiniuj warunki połączeń i filtrowania w \rCode{WHERE}.
    \item Dodaj \rCode{ORDER BY}, aby sprawdzić dane przed grupowaniem.
    \item Dodaj \rCode{GROUP BY} z odpowiednimi funkcjami agregującymi.
    \item Użyj \rCode{HAVING}, jeśli chcesz odfiltrować całe grupy.
    \item Na końcu określ kolejność wyników za pomocą \rCode{ORDER BY}.
\end{enumerate}

---

