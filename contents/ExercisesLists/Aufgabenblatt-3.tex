% filename Aufgabenblatt-3.tex
% !TeX root = ../../main.tex
%%%%%

\section{Aufgabenblatt 3}

\subsubsection{Geben Sie aus, wie viele Module in der Datenbank gespeichert sind.}
    \begin{sql}
        SELECT COUNT(*)
            FROM modul      
    \end{sql}

    \subsubsection{Geben Sie alle Module (Name) aus, die weniger als 4 ECTS-Punkte haben.}
    \begin{sql}
        SELECT name
            FROM modul
            WHERE ects < 4
    \end{sql}

    \subsubsection{Geben Sie alle Module aus, in denen eine Klausur als Prüfung möglich ist.}
    \begin{sql}
        SELECT modul.name
            FROM modul
            WHERE modul.pruefung like '%Klausur%'
    \end{sql}

    \subsubsection{Geben Sie die Namen der Module aus, bei denen als Modulverantwortlicher "F. Rump" angegeben ist.}
    \begin{sql}
        SELECT DISTINCT modul.name
            FROM modul, person
            WHERE modul.pid = person.pid AND person.name = 'F. Rump'

                -- oder

        SELECT DISTINCT modul.name
            FROM modul
            JOIN person ON modul.pid = person.pid
            WHERE person.name = 'F. Rump'
    \end{sql}

    \subsubsection{eben Sie alle Studiengänge mit den zugehörigen Zertifikaten aus (Ausgabe: Studiengang, Zertifikat).}
    \begin{sql}
        SELECT DISTINCT studiengang.name, zertifikat.name
            FROM zertifikatzuordnung
            JOIN zertifikat ON zertifikat.zid = zertifikatzuordnung.zid
            JOIN studiengang ON studiengang.sid = zertifikatzuordnung.sid

            -- oder

        SELECT DISTINCT studiengang.name Modulname, zertifikat.name Zertifikatname
            FROM zertifikat, zertifikatzuordnung, studiengang
            WHERE studiengang.sid = zertifikatzuordnung.sid
            AND zertifikatzuordnung.zid = zertifikat.zid
    \end{sql}

    \subsubsection{Geben Sie alle Lehrenden und die Veranstaltungen aus, die sie unterrichten.}
    \begin{sql}
        SELECT DISTINCT person.name, veranstaltung.name
            FROM lehrende
            JOIN person ON person.pid = lehrende.pid
            JOIN veranstaltung ON veranstaltung.vid = lehrende.vid

            -- oder

        SELECT DISTINCT person.name, veranstaltung.name
            FROM lehrende, person, veranstaltung
            WHERE lehrende.pid = person.pid AND lehrende.vid = veranstaltung.vid
    \end{sql}

    \subsubsection{Geben Sie alle Module aus, sofern die enthaltenen Veranstaltung in Summe mehr als vier SWS haben}
    \begin{sql}
        SELECT DISTINCT modul.name
            FROM modul, veranstaltung
            WHERE modul.mid = veranstaltung.mid
            AND veranstaltung.sws > 4
    \end{sql}

    \subsubsection{Geben Sie nur die Veranstaltungen aus, denen mehrere Lehrende zugeordnet sind.}
    \begin{sql}
        SELECT veranstaltung.name
            FROM veranstaltung, lehrende
            WHERE veranstaltung.vid = lehrende.vid
            GROUP BY veranstaltung.name
            HAVING COUNT(DISTINCT lehrende.pid) > 1
    \end{sql}

    \subsubsection{Geben Sie alle Studiengänge absteigend nach der Anzahl der "reinen" Wahlpflichtmodule, die somit keinem Zertifikat zugeordnet sind, aus.}
    \begin{sql}
        SELECT studiengang.name, COUNT(*) count
            FROM studiengang, modulzuordnung
            WHERE studiengang.sid = modulzuordnung.sid
            AND modulzuordnung.semester IS NULL
            GROUP BY studiengang.name
            ORDER BY COUNT(*) desc
    \end{sql}
