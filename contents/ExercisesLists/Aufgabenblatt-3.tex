% filename Aufgabenblatt-3.tex
% !TeX root = ../../main.tex
%%%%%

\newpage
\section{Aufgabenblatt 3}

\subsubsection{Geben Sie aus, wie viele Module in der Datenbank gespeichert sind.}
    \begin{sql}
        SELECT COUNT(*)
            FROM modul      
    \end{sql}

    \subsubsection{Geben Sie alle Module (Name) aus, die weniger als 4 ECTS-Punkte haben.}
    \begin{sql}
        SELECT name
            FROM modul
            WHERE ects < 4
    \end{sql}

    \subsubsection{Geben Sie alle Module aus, in denen eine Klausur als Prüfung möglich ist.}
    \begin{sql}
        SELECT modul.name
            FROM modul
            WHERE modul.pruefung like '%Klausur%'
    \end{sql}

    \subsubsection{Geben Sie die Namen der Module aus, bei denen als Modulverantwortlicher "F. Rump" angegeben ist.}
    \begin{sql}
        SELECT DISTINCT modul.name
            FROM modul, person
            WHERE modul.pid = person.pid AND person.name = 'F. Rump'

                -- oder

        SELECT DISTINCT modul.name
            FROM modul
            JOIN person ON modul.pid = person.pid
            WHERE person.name = 'F. Rump'
    \end{sql}

    \subsubsection{eben Sie alle Studiengänge mit den zugehörigen Zertifikaten aus (Ausgabe: Studiengang, Zertifikat).}
    \begin{sql}
        SELECT DISTINCT studiengang.name, zertifikat.name
            FROM zertifikatzuordnung
            JOIN zertifikat ON zertifikat.zid = zertifikatzuordnung.zid
            JOIN studiengang ON studiengang.sid = zertifikatzuordnung.sid

            -- oder

        SELECT DISTINCT studiengang.name Modulname, zertifikat.name Zertifikatname
            FROM zertifikat, zertifikatzuordnung, studiengang
            WHERE studiengang.sid = zertifikatzuordnung.sid
            AND zertifikatzuordnung.zid = zertifikat.zid
    \end{sql}

    \subsubsection{Geben Sie alle Lehrenden und die Veranstaltungen aus, die sie unterrichten.}
    \begin{sql}
        SELECT DISTINCT person.name, veranstaltung.name
            FROM lehrende
            JOIN person ON person.pid = lehrende.pid
            JOIN veranstaltung ON veranstaltung.vid = lehrende.vid

            -- oder

        SELECT DISTINCT person.name, veranstaltung.name
            FROM lehrende, person, veranstaltung
            WHERE lehrende.pid = person.pid AND lehrende.vid = veranstaltung.vid
    \end{sql}

    \subsubsection{Geben Sie alle Module aus, sofern die enthaltenen Veranstaltung in Summe mehr als vier SWS haben}
    \begin{sql}
        SELECT modul.name, SUM(veranstaltung.sws)
            FROM veranstaltung, modul
            WHERE veranstaltung.mid = modul.mid
            GROUP BY modul.mid
            HAVING SUM(veranstaltung.sws) > 4
    \end{sql}

    \subsubsection{Geben Sie nur die Veranstaltungen aus, denen mehrere Lehrende zugeordnet sind.}
    \begin{sql}
        SELECT veranstaltung.name
            FROM veranstaltung, lehrende
            WHERE veranstaltung.vid = lehrende.vid
            GROUP BY veranstaltung.name
            HAVING COUNT(DISTINCT lehrende.pid) > 1
    \end{sql}

    \subsubsection{Geben Sie alle Studiengänge absteigend nach der Anzahl der "reinen" Wahlpflichtmodule, die somit keinem Zertifikat zugeordnet sind, aus.}
    \begin{sql}
        SELECT studiengang.name
            FROM studiengang, modulzuordnung
            WHERE studiengang.sid = modulzuordnung.sid
            AND modulzuordnung.semester IS NULL
            GROUP BY studiengang.name
            ORDER BY COUNT(*) desc
    \end{sql}

    \subsubsection{Geben Sie zu jedem Studiengang und jedem Zertifikat in dem jeweiligen Studiengang die Anzahl der zugehörigen Module aus, wobei die Studiengänge und Zertifikate alphabetisch sortiert sein sollen.}
    \begin{sql}
        SELECT studiengang.name, zertifikat.name, COUNT(*) moduleCnt
            FROM studiengang, zertifikatzuordnung, zertifikat, zertifikatsmodul, modul
            WHERE studiengang.sid = zertifikatzuordnung.sid 
            AND zertifikatzuordnung.zid = zertifikat.zid
            AND zertifikatsmodul.zid = zertifikat.zid
            AND modul.mid = zertifikatsmodul.mid
            GROUP BY studiengang.name, zertifikat.name
            ORDER BY studiengang.name, zertifikat.name ASC
    \end{sql}

    \subsubsection{Geben Sie zu jedem Studiengang die Anzahl der Zertifikate aus, sofern der Studiengang mehr als drei Zertifikate hat.}
    \begin{sql}
        SELECT studiengang.name, count(*) zertifikatCnt
            FROM studiengang, zertifikatzuordnung, zertifikat
            WHERE studiengang.sid = zertifikatzuordnung.sid
            AND zertifikatzuordnung.zid = zertifikat.zid
            GROUP BY studiengang.name
            HAVING COUNT(*) > 3
    \end{sql}

    \subsubsection{Geben Sie die Studiengänge absteigend nach den daran beteiligten, unterschiedlichen Lehrenden sortiert aus.}
    \begin{sql}
        SELECT studiengang.name, count(distinct lehrende.pid) lehrendeCnt
            FROM studiengang, modulzuordnung, veranstaltung, lehrende
            WHERE studiengang.sid = modulzuordnung.sid
            AND modulzuordnung.mid = veranstaltung.mid
            AND veranstaltung.vid = lehrende.vid
            GROUP BY studiengang.name
            ORDER BY count(distinct lehrende.pid) DESC
    \end{sql}


    \subsubsection{Geben Sie für alle Lehrenden aus, wie viele Veranstaltungen sie lehren.}
    \begin{sql}
        SELECT person.name, count(*) veranstaltungCnt
            FROM person, lehrende, veranstaltung
            WHERE person.pid = lehrende.pid
            AND lehrende.vid = veranstaltung.vid
            GROUP BY person.name
    \end{sql}

    \subsubsection{Geben Sie die Lehrbeauftragten aus, die nur genau eine Veranstaltung anbieten.}
    \begin{sql}
        SELECT person.name
            FROM person, lehrende, veranstaltung
            WHERE person.pid = lehrende.pid
            AND lehrende.vid = veranstaltung.vid
            AND lehrbeauftragter = true
            GROUP BY person.name
            HAVING COUNT(*) = 1
    \end{sql}

    \subsubsection{Geben Sie für jeden Studiengang die Semester (sofern ungleich NULL) aufsteigend sortiert mit der Summe der ECTS der enthaltenen Module an.}
    \begin{sql}
        SELECT studiengang.name, modulzuordnung.semester, SUM(modul.ects)
            FROM studiengang, modulzuordnung, modul
            WHERE studiengang.sid = modulzuordnung.sid
            AND modulzuordnung.mid = modul.mid
            AND modulzuordnung.semester IS NOT NULL
            GROUP BY studiengang.name, modulzuordnung.semester
            ORDER BY studiengang.name, modulzuordnung.semester
    \end{sql}
