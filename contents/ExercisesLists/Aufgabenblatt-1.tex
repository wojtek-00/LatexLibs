% filename.....
% !TeX root = ../main.tex
%%%%%

\section{Aufgabenblatt 1}
\subsection{Aufgabe 1}
Der Ren-Operator wird benötigt, wenn ein Kreuzprodukt einer Tabelle mit sich selbst (Self-Join) gebildet werden muss.

    \begin{table}[H]
        \centering
        \textbf{Aufgaben:} \\ [10pt]
        \begin{tabular}{|c|c|c|}
            \hline
            \textbf{ProzessNr} & \textbf{Name} & \textbf{VorgängerNr} \\
            \hline
            1 & Schneiden   & - \\
            2 & Waschen     & 1 \\
            3 & Biegen      & 2 \\
            4 & Bohren      & 2 \\
            5 & Malen       & 4\\
            \hline
        \end{tabular}
    \end{table}

    Gib die Aufgaben und deren Vorgänger aus.
    \begin{align*}
        \text{Proj}\bigl(
        \text{Sel}&(Aufgaben \times Ren(Aufgaben, A2),\, Aufgabe.VorgängerNr = A2.ProzessNr),\, \\
        &\text{[Aufgabe.Name, A2.Name]}
        \bigr)
    \end{align*}


\subsection{Aufgabe 2}
\begin{table}[H]
    \scriptsize
    \centering
    \textbf{Relationen} \\ [10pt]
    \begin{minipage}[t]{0.2\textwidth}
        \centering
        \textbf{Student} \\[3pt]
        \begin{tabular}{|c|c|}
            \hline
            \textbf{MatNr} & \textbf{Name} \\
            \hline
            1 & Meier \\
            2 & Meyer \\
            3 & Maier \\
            \hline
        \end{tabular}
    \end{minipage}
    \hspace{0.2cm}
    \begin{minipage}[t]{0.3\textwidth}
        \centering
        \textbf{Gericht} \\ [3pt]
        \begin{tabular}{|c|c|c|}
            \hline
            \textbf{GNr} & \textbf{Name} & \textbf{Art} \\
            \hline
            1 & Pizza            & Haupt \\
            2 & TomatenSuppe     & Vor \\
            3 & Schnitzel        & Haupt \\
            4 & Reis             & Beilage \\
            5 & Pudding          & Nach \\
            \hline
        \end{tabular}
    \end{minipage}
    \hspace{0.2cm}
    \begin{minipage}[t]{0.3\textwidth}
        \centering
        \textbf{Bewertung} \\ [3pt]
        \begin{tabular}{|c|c|c|}
            \hline
            \textbf{MatrNr} & \textbf{GNr} & \textbf{Sterne} \\
            \hline
            1 & 2    & 3 \\
            1 & 4    & 2 \\
            2 & 1    & 4\\
            3 & 3    & 3 \\
            \hline
        \end{tabular}
    \end{minipage}
\end{table}

\begin{enumerate}
    \item Geben Sie alle Arten von Gerichten aus.
        \begin{align*}
            \text{Proj}(Gericht,\, \text{[Art]})
        \end{align*}

        \begin{table}[H]
            \centering
            \textbf{Ergebnisstabelle} \\ [3pt]
            \begin{tabular}{|c|}
                \hline
                \textbf{Art}\\
                \hline
                Haupt \\
                Vor \\
                Beilage \\
                Nach\\
                \hline  
            \end{tabular}
        \end{table}

    \item Geben Sie die Namen aller Hauptgerichte (mit der Art „Haupt“) aus.
        \begin{align*}
            \text{Proj}\bigl(
            \text{Sel}(Gericht,\,[Art=\text{'Haupt'}]),\, \text{[Name]}
            \bigr)
        \end{align*}

        \begin{table}[H]
            \centering
            \textbf{Ergebnisstabelle} \\ [3pt]
            \begin{tabular}{|c|}
                \hline
                \textbf{Name}\\
                \hline
                Pizza \\
                Schnitzel \\
                \hline  
            \end{tabular}
        \end{table}

    \item Geben Sie eine Liste aller einzelnen Bewertungen aus (Ausgabe: Name des Gerichts, Sterne).
        \begin{align*}
            \text{Proj}\bigl(
            \text{Sel}(Gericht \times Bewertung,\, Gericht.GNr = Bewertung.GNr),\, \text{[Name, Sterne]}
            \bigr)
        \end{align*}

        \begin{table}[H]
            \centering
            \textbf{Ergebnisstabelle} \\ [3pt]
            \begin{tabular}{|c|c|}
                \hline
                \textbf{Name} & \textbf{Sterne}\\
                \hline
                Pizza & 4 \\
                TomatenSuppe & 3 \\
                Schnitzel & 3 \\
                Reis & 2 \\
                \hline  
            \end{tabular}
        \end{table}

    \item Geben Sie die Namen aller Gerichte aus, die der Student Meier bewertet hat.
        \begin{align*}
            \text{Proj}\bigl(&
            \text{Sel}(Student \times Gericht \times Bewertung,\, Student.MatNr = Bewertung.MatNr \\
            &\text{ AND } Bewertung.GNr = Gericht.GNr \text{ AND } Student.Name=\text{'Meier'}),\, \\
            &\text{[Gericht.Name]}
            \bigr)
        \end{align*}

        \begin{table}[H]
            \centering
            \textbf{Ergebnisstabelle} \\ [3pt]
            \begin{tabular}{|c|}
                \hline
                \textbf{Name}\\
                \hline
                TomatenSuppe \\
                Reis \\
                \hline  
            \end{tabular}
        \end{table}

    \item Geben Sie alle Bewertungen aus (Name Student, Name Gericht, Sterne), die mindestens vier Sterne haben.
        \begin{align*}
            \text{Proj}&\bigl(
            \text{Sel}(Student \times Gericht \times Bewertung,\, Student.MatNr=Bewertung.MatrNr \text{ AND }\\
            &Bewertung.GNr=Gericht.GNr \text{ AND } Bewertung.Sterne\text{ >= } 4),\\
            &\text{[Student.Name, Gericht.Name, Sterne]}
            \bigr)
        \end{align*}

        \begin{table}[H]
            \centering
            \textbf{Ergebnisstabelle} \\ [3pt]
            \begin{tabular}{|c|c|c|}
                \hline
                \textbf{Name Student} & \textbf{Name Gericht} & \textbf{Sterne}\\
                \hline
                Meyer & Pizza & 4 \\
                \hline  
            \end{tabular}
        \end{table}

    \item Geben Sie aus, welche Studierenden das Schnitzel bewertet haben.
        \begin{align*}
            \text{Proj}&\bigl(
            \text{Sel}(Student \times Gericht \times Bewertung,\,
            \text{Student.MatNr} = \text{Bewertung.MatNr} \text{ AND }\\
            &\text{Bewertung.GNr} = \text{Gericht.GNr} \text{ AND }
            \text{Gericht.Name} = \text{'Schnitzel'}),
            [\text{Student.Name}]
            \bigr)
            \end{align*}

        \begin{table}[H]
            \centering
            \textbf{Ergebnisstabelle} \\ [3pt]
            \begin{tabular}{|c|}
                \hline
                \textbf{Student Name}\\
                \hline
                Maier \\
                \hline  
            \end{tabular}
        \end{table}
        
    \item Geben Sie aus, welcher Studierende mindestens zwei Bewertungen abgegeben hat.
        \begin{align*}
            \text{Proj}&\bigl(
            \text{Sel}(Student \times Bewertung \times \text{Ren}(Bewertung,\, B2), \\ 
            &Student.MatrNr=Bewertung.MatrNr\\
            &\text{ AND } Student.MatrNr = B2.MatrNr\\
            & \text{ AND } Bewertung.GNr \text{ <> } B2.GNr),\, [Student.Name]
            \bigr)
        \end{align*}

        \begin{table}[H]
            \centering
            \textbf{Ergebnisstabelle} \\ [3pt]
            \begin{tabular}{|c|}
                \hline
                \textbf{Student Name}\\
                \hline
                Meier \\
                \hline  
            \end{tabular}
        \end{table}

\end{enumerate}
