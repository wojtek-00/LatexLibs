% filename.....
% !TeX root = ../main.tex
%%%%%

\section{Aufgabenblatt 2}
\begin{table}[H]
    \scriptsize
    \centering
    \textbf{Relationen} \\ [10pt]
    \begin{minipage}[t]{0.2\textwidth}
        \centering
        \textbf{Student} \\[3pt]
        \begin{tabular}{|c|c|}
            \hline
            \textbf{MatNr} & \textbf{Name} \\
            \hline
            1 & Meier \\
            2 & Meyer \\
            3 & Maier \\
            \hline
        \end{tabular}
    \end{minipage}
    \hspace{0.2cm}
    \begin{minipage}[t]{0.3\textwidth}
        \centering
        \textbf{Klausur} \\ [3pt]
        \begin{tabular}{|c|c|c|c|}
            \hline
            \textbf{KNr} & \textbf{Name} & \textbf{Datum} & \text{Zeit} \\
            \hline
            1 & Java 1              & 2024-01-14 & 10:00:00\\
            2 & Einführung Inf.     & 2024-01-16 & 08:00:00 \\
            3 & Mathematik 1        & 2024-01-20 & 13:00:00\\
            4 & Medieninformatik    & 2024-01-20 & 08:00:00\\
            5 & Audio/Video         & 2024-01-28 & 15:30:00 \\
            \hline
        \end{tabular}
    \end{minipage}
    \newline \\ [10pt]
    \begin{minipage}[t]{0.3\textwidth}
        \centering
        \textbf{Bewertung} \\ [3pt]
        \begin{tabular}{|c|c|c|}
            \hline
            \textbf{MatrNr} & \textbf{KNr} & \textbf{Versuch} \\
            \hline
            1 & 2    & 1 \\
            1 & 4    & 2 \\
            2 & 1    & 2\\
            3 & 3    & 3 \\
            \hline
        \end{tabular}
    \end{minipage}
\end{table}

\subsection{Aufgabe 2.1}
    \begin{lstlisting}[style=sqlstyle, caption={Erstellen der Tabellen STUDENT, KLAUSUR und ANMELDUNG}]
    CREATE TABLE STUDENT(
        MatrNr INTEGER,
        Name VARCHAR(5),
        PRIMARY KEY (MatrNr)
    );

    CREATE TABLE KLAUSUR(
        KNr INTEGER,
        Name VARCHAR(25),
        Datum DATE,
        Zeit TIME,
        PRIMARY KEY (KNr)
    );

    CREATE TABLE ANMELDUNG(
        MatrNr INTEGER,
        KNr INTEGER,
        Versuch INTEGER,
        CONSTRAINT FK_MatrNr
            FOREIGN KEY (MatrNr)
            REFERENCES STUDENT(MatrNr),
        CONSTRAINT FK_KNr
            FOREIGN KEY (KNr)
            REFERENCES KLAUSUR(KNr)
    );
    \end{lstlisting}
    
    \vspace{1cm}
    \begin{lstlisting}[style=sqlstyle, caption={Einfügen von Datensätzen in die Tabellen STUDENT, KLAUSUR und ANMELDUNG}]
    INSERT INTO STUDENT VALUES(1,'Meier');
    INSERT INTO STUDENT VALUES(2,'Meyer');
    INSERT INTO STUDENT VALUES(3,'Maier');

    INSERT INTO KLAUSUR VALUES(1,'Java 1','2024-01-14', '10:00:00');
    INSERT INTO KLAUSUR VALUES(2,'Einfuhrung Inf.','2024-01-16', '08:00:00');
    INSERT INTO KLAUSUR VALUES(3,'Mathematik 1','2024-01-20', '13:00:00');
    INSERT INTO KLAUSUR VALUES(4,'Medieninformatik','2024-01-20', '08:00:00');
    INSERT INTO KLAUSUR VALUES(5,'Audio/Video','2024-01-28', '15:30:00');

    INSERT INTO ANMELDUNG VALUES(1,2,1);
    INSERT INTO ANMELDUNG VALUES(1,4,2);
    INSERT INTO ANMELDUNG VALUES(2,1,2);
    INSERT INTO ANMELDUNG VALUES(3,3,3);
    \end{lstlisting}

    \subsection{Aufgabe 2.2}
    \begin{lstlisting}[style=sqlstyle, caption={Einfügen von Datensätzen in die Tabellen STUDENT, KLAUSUR und ANMELDUNG}]
    -- 1. Geben Sie die Namen aller Studierenden aus.
    SELECT klausur.Name FROM student;
    
    -- 2. Geben Sie die Namen aller Klausuren aus, die um 08:00 Uhr geschrieben werden.
    SELECT klausur.name FROM klausur WHERE zeit = '08:00:00';

    -- 3. Geben Sie eine Liste aller Erstanmeldungen (nur 1. Versuch) fuer eine Klausur aus (Ausgabe: Name der Klausur, Name des Studierenden).
    SELECT klausur.name, student.name FROM klausur, student, anmeldung WHERE anmeldung.versuch = 1 AND klausur.knr = anmeldung.knr AND student.matrnr = anmeldung.matrnr;

    -- 4. Geben Sie die Namen aller Klausuren aus, feur die sich die Studentin "Meier" angemeldet hat.
    SELECT klausur.name FROM klausur, student, anmeldung WHERE student.name Like 'Meier' AND klausur.knr = anmeldung.knr AND student.matrnr = anmeldung.matrnr;

        -- oder

    SELECT klausur.name FROM student, klausur, anmeldung WHERE anmeldung.matrnr = student.matrnr AND anmeldung.knr = klausur.knr AND student.name='Meier';

    -- 5. Geben Sie die Namen aller Studierenden aus, die mindestens zwei Klausuren im letzten Versuch (3. Versuch) schreiben.
    SELECT DISTINCT student.name
        FROM student, anmeldung a1, anmeldung a2
        WHERE student.matrnr = a1.matrnr
            AND student.matrnr = a2.matrnr
            AND a1.knr <> a2.knr
            AND a1.versuch >= 3
            AND a2.versuch >= 3;

        -- oder
        
    SELECT DISTINCT name FROM student
        JOIN anmeldung AS a1 ON a1.matrnr=student.matrnr
        JOIN anmeldung AS a2 ON a2.matrnr=student.matrnr
        WHERE a1.versuch >= 3
            AND a2.versuch >= 3
            AND a1.knr <> a2.knr;   
    \end{lstlisting}
    